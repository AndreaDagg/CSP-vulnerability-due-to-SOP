\section{INTRODUZIONE}

\justifying

All'interno di una comunicazione client-server, abbiamo che nei moderni browser può essere eseguito del contenuto dinamicamente, andando ad esempio a sfruttare quella che è la potenzialità offerta dai linguaggi di scripting come javascript. Ne consegue che all'interno dei browser possono essere memorizzate delle informazioni sensibili da proteggere dell'accesso di terzi non autorizzati, come ad esempio i cookie che possono contenere credenziali d'accesso e/o token d'autenticazione. A tale scopo sono state sviluppate delle policy di sicurezza per limitare (Same Origin Policy) e controllare (Content Security Policy) l'utilizzo e la gestione delle risorse all'interno delle pagine web riducendo i rischi di poter effettuare l'injection di codice malevolo. Tuttavia la policy CSP consegnata con una pagina web influisce solo sulle risorse della pagina stessa, non estendendosi automaticamente alle risorse incorporate nella pagina, come possono essere i tag html $<iframe>$. Questo comporta che anche nel caso in cui la pagina principale possa avere una politica CSP le regole non si applicano al contenuto all'interno degli iframe.
La diretta conseguenza è che, sfruttando l'origine in comune, si aprono le porte a diverse problematiche di sicurezza, consentendo l'esecuzione di codice arbitrario, raggirando quindi la Same Origin Policy (SOP), ma anche la possibilità di poter definire policy CSP differenti tra pagine importatrici e pagine importate tramite iframe, avendo in conclusione un'inconsistenza di regole definite nelle diverse policy. Verranno quindi individuate e replicate diverse casistiche per valutare se questa vulnerabilità è tutt'oggi una minaccia e valutare la possibilità di poter mitigare l'attuazione di tali scenari. 