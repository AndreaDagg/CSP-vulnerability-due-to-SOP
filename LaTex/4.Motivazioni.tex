\section{Motivazioni}
\justifying 
Definita una policy CSP all'interno di una pagina Web questa si applica solamente alle risorse inerenti la pagina stessa, di conseguenza non viene estesa ad elementi incorporati all'interno del codice come gli iframe. Nello specifico, considerando una web-page nella quale si utilizza un tag iframe con l'obiettivo di importare una seconda webpage \underline{appartenente alla stessa origine}, si può incorrere nelle seguenti casistiche:\\
\textbf{- Solamente la pagina importatrice dell'iframe ha definita una policy CSP:} In questo caso il codice contenuto nell'iframe raggirerà sia SOP che la policy CSP. Un attacco XSS è possibile in quando il codice contenuto nell'iframe è potenzialmente vulnerabile e per via dell'inconsistenza tra SOP e CSP questa vulnerabilità si estenderà al codice della pagina potendo accedere al codice della pagina principale.\\
\textbf{- Solamente il codice all'interno dell'iframe ha definita una policy CSP:} Questo comporta che la pagina principale sia vulnerabile ad eventuali attacchi XSS ed importando un iframe in cui, è si definita una policy CSP, ma che provenendo dalla stessa origine raggiri SOP, consente all'eventuale codice dannoso presente nella pagina principale di accedere al codice dell'iframe. \\

Un ulteriore caso è quello che riguarda la comunicazione tra la pagina principale (main.com) ed una pagina appartenente ad un suo sotto-dominio (sub.main.com). Benché condividano lo stesso dominio di alto livello, SOP valuta queste due origini differenti andando a bloccarne la comunicazione. Tuttavia quello che può accadere è che utilizzando la direttiva \textit{document.domain} la pagina importata (sub.main.com) può cambiare il suo dominio, rilassandolo al dominio di primo livello diventando di fatto "main.com". in questo modo sia nel caso in cui la pagina principale o la pagina all'interno dell'iframe non abbiano una policy CSP implementata, oppure nel caso abbiano due policy CSP diverse, è possibile raggirare SOP rendendo possibile  gli attacchi precedentemente descritti. 
%Scoperto che doc.domain è deprecato va nei risultati ;D