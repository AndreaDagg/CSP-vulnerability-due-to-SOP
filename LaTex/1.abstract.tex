\renewcommand{\abstractname}{\textcolor{carnelian}{ABSTRACT}}
\begin{abstract}
\vspace*{0.5cm}
\fontsize{11pt}{11pt}\selectfont
\justify

Nell’ambito delle Web Application, più nello specifico nei moderni browser, per mitigare il rischio di violazioni, sono implementate delle policy di sicurezza. Due tra le principali sono la \textbf{Same Origin Policy (SOP)} e la \textbf{Content Security Policy (CSP)}. Entrambe le policy sono dei meccanismi atti a regolare il corretto caricamento di risorse all'interno delle pagine web, quali, per citarne alcuni, file di scripting js, di immagini e di style.\\
Questo lavoro mira a descrivere, attraverso varie casistiche, come la Content Security Policy (CSP) possa essere violata a causa delle Same Origin Policy (SOP) nel caso in cui una web-page contenga un i-frame embedded appartenente alla stessa \textbf{origine} e ad individuare un meccanismo di mitigazione di questa vulnerabilità. 

\keywordsEng{Same Origin Policy, Content Security Policy, Web Application violation}

\end{abstract}
\vspace*{0.3cm}