\section{AMBIENTE}
\justifying
Per replicare lo studio è stata necessaria la creazione di due server distinti. Questi server hanno il compito di dover eseguire le varie pagine web nelle quali sono definite le diverse direttive atte a replicare le casistiche da studiare. Per rispondere alle due domande proposte nel capitolo \ref{chap:obiettivi}, i server devono essere eseguibili su due ambienti distinti:
\begin{itemize}
\item  l'ambiente odierno;
\item  l'ambiente obsoleto;
\end{itemize}
 Questo con lo scopo di replicare lo studio e valutarne l'eventuale mitigazione nel corso degli anni. Per assolvere a questi scopi si è utilizzata la tecnologia Docker in quanto consente di rendere disponibile lo stesso applicativo su molteplici ambienti isolati e differenti. 
\subsection{Docker server}
I server sono stati realizzati utilizzando Apache HTTP Server\footnote{\url{https://httpd.apache.org/}} nello specifico, per quanto riguarda il primo server, \textit{ServerOne}, nello snippet Listing \ref{snippet:dockerComposeS1} è riportato il codice del docker-compose, mentre nello snippet Listing\ref{snippet:dockerFile} viene riportato il codice relativo il Dockerfile.
\lstinputlisting[style=yaml,
  caption=Server One docker-compose,
    label=snippet:dockerComposeS1
]{Resources/ServerOne/docker-compose.yml} 
In questo modo andiamo a definire come porta assegnata al container del server la porta 8001. Per quanto riguarda il Server Two, il file docker-compose sarà simile con l'unica differenza della definizione della porta su 8002.
\lstinputlisting[style=yaml,
  caption={Server One Dockerfile},
  label={snippet:dockerFile}
]{Resources/ServerOne/Dockerfile}
Utilizziamo il Dockerfile per definire la versione del server Apache e solamente per il Server One la definizione dei file di configurazione per la creazione del sottodominio. 

\subsection{Sottodominio}
È stato poi necessario creare un sottodominio all'interno del Server One con lo scopo di voler dimostrare che la Same Origin Policy bloccasse le richieste provenienti da questo sottodominio verso il dominio principale e che quindi effettivamente i due domini \textit{hostone.dagg} e \textit{sub.hostone.dagg} venissero considerati come due domini differenti. Successivamente andare a studiare l'utilizzo della direttiva \textit{document.domain} per raggirare questo controllo su ambiente obsoleto e accertarsi della mitigazione sull'ambiente odierno essendo ad oggi document.domain una direttiva deprecata \cite{5}.

\begin{table*}
    \centering
    \caption{Elementi dei server, pagine, domini e porte}
    \medskip
    %\resizebox{0.7\linewidth}{!}{
        \begin{tabular}{|c|c|c|c|c|}
        \hline
        \rowcolor{gray!30} % Colora l'intera riga con il 50% del grigio
            \textbf{Server Name}         & \textbf{Port}        & \textbf{Domain}  &\textbf{subdomian} &\textbf{server pages} \\
            \hline
            Server One                      &   8001                      & hostone.dagg:8001 & sub.hostone.dagg:8001 & \begin{tabular}[m]{@{}c@{}}
                                                                                                                            a.html \\
                                                                                                                            b.html \\
                                                                                                                            c.html \\
                                                                                                                            d.html \\
                                                                                                                            e.html \\
                                                                                                                            sub.html
                                                                                                                    \end{tabular}  \\ 
            \hline
            Server Two                      &   8002                      & hostone.dagg:8002 & \null & index.html \\
            \hline
        \end{tabular}
    %}
    \label{tab:ServerOneAndTwo}
\end{table*}


\subsubsection{Dominio e DNS}
Per la creazione di un sottodominio nel Server One bisognava partire dalla definizione di un dominio associato al server docker eseguito su localhost 127.0.0.1:8001. Definito quindi il dominio personale \textit{hostone.dagg}, per configurare un sottodominio bisognava passare da un dominio di secondo livello ad un dominio di terzo livello \textit{sub.hostone.dagg}. Per realizzare questa configurazione tramite un server Apache si è realizzato un file \textit{subDomain.conf}, Listing\ref{snippet:subdomainconf}, andando a specificare un nuovo Virtual Host per il server Host One che serve i file nella directory /var/www/html/sub quando riceve richieste per l'hostname sub.hostone.dagg:8001 sulla porta 80. 
\lstinputlisting[style=yaml,
    caption= {File configurazione sotto dominio},
    label={snippet:subdomainconf}
]{Resources/ServerOne/subDomain.conf} 
Come descritto nel Dockerfile, Listing\ref{snippet:dockerFile}, questo file di configurazione viene copiato nella directory di configurazione del server Apache \textit{../sites-available/} e viene poi riavviato il server per rendere le modifiche persistenti eseguendo i comandi: 
\begin{itemize}
    \item \small\textit{COPY sub/subDomain.conf /etc/apache2/sites-available/}
    \item \small\textit{RUN a2ensite subDomain.conf \&\& service apache2 restart}
\end{itemize}
%### Alla fine puntare alla tabella riassuntiva
Eseguendo il server in rete locale, per rendere effettivo il raggiungimento del server tramite un dominio personale come \textit{hostone.dagg:8001}, quindi non utilizzado i classici indirizzi \textit{localhost} o \textit{127.0.0.1}, bisogna andare a fornire istruzioni nel DNS\footnote{Domain Name System} locale dei sistemi operativi su come tradurre tali domini in indirizzi IP. Per farlo si va a modificare il già presente file di configurazione \textit{../hosts}. Raggiungibile nello specifico a seconda dell'ambiente attraverso le path: 
\begin{itemize}
    \item \textbf{Windows:} \textit{C:\textbackslash Windows\textbackslash System32\textbackslash drivers\textbackslash etc\textbackslash hosts}
    \item \textbf{Ubuntu:} \textit{/etc/hosts}
\end{itemize}
All'interno del file \textit{hosts} in figura listing\ref{dns_host_file} sono state aggiunte le righe relative la traduzione del nome del dominio personale, definito nei file di configurazione del web server, in indirizzo IP locale \textit{127.0.0.1}. 

\begin{lstlisting}[
    style=HTML,
    caption={File hosts del DNS locale},
    label={dns_host_file}
]
                    ...

# localhost name resolution
#	127.0.0.1       localhost

# Custom definition for personal domain
127.0.0.1 hostone.dagg 
127.0.0.1 sub.hostone.dagg 

                    ...
\end{lstlisting}

In questo modo è stato possibile raggiungere i web server dockerizzati attraverso il dominio personale \textit{hostone.dagg}, andando a specificare il numero di porta per discriminare il "Server One" dal "Server Two" e raggiungendo il sotto dominio di "Server One" attraverso il dominio \textit{sub.hostone.dagg:8001}. Per avere una visione d'insieme più organizzata dei server e sulle pagine che appartengono ad ogni server, in Table\ref{tab:ServerOneAndTwo} è possibile osservare una struttura organizzata. 

\subsection{Ambiente Obsoleto}
Con l'obiettivo di voler replicare lo studio è stato necessario eseguire gli esperimenti su un ambiente obsoleto. Nel particolare in riferimento al lavoro \cite{1}, si sono individuate le versioni di \textbf{Ubuntu 16.04} \footnote{ Disponibile al link: https://releases.ubuntu.com/16.04/)} e del browser \textbf{Google Chrome nella versione v.51-0-2701} \footnote{ Disponibile al link: https://www.slimjet.com/chrome/google-chrome-old-version.php)}. Per configurare l'ambiente si è quindi creata una macchina virtuale tramite il software di virtualizzazione Virtual Box\footnote{https://www.virtualbox.org/}. Trattandosi di versioni obsolete per una maggiore sicurezza la macchina virtuale è stata eseguita in ambiente di sandboxing. L'obiettivo di tale configurazione è appunto quello di avere un ambiente quanto più possibile simile all'ambiente in cui è stato svolto il lavoro in analisi\cite{1} in modo da poter valutare le casistiche sulle versioni delle policy SOP e CSP negli anni in cui il lavoro è stato sottomesso. 