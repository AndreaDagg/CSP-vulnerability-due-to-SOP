\section{Metodologia e Deliverables}
\justifying
Come primo step per la replicazione degli attacchi, si procederà andando ad individuare tutte le casistiche possibili. Successivamente bisognerà creare una environment virtualizzato che risponda alle specifiche descritte nel lavoro \cite{1}. 
Per la creazione dei web server e per la specifica dei differenti domini necessari alla sperimentazione si userà la tecnologia \href{https://www.docker.com/}{Docker}. Il codice sviluppato andrà ad effettuare l'injection e raggirando le policy accederà a delle specifiche variabili. Si procederà infine alla ricerca di eventuali strategie di mitigazioni valutando la possibilità di poterle attuare.

\subsection{Deliverables}
Alla fine della sperimentazione si presenteranno i risultati relativi la riuscita o meno della replicazione degli attacchi ed i dati relativi la possibilità di attuare o meno delle mitigazioni per la tipologia d'attacco in esame. Il codice sviluppato sarà disponibile all'interno della repository GitHub: 
\begin{center}
\url{https://github.com/AndreaDagg/CSP-vulnerability-due-to-SOP}
\end{center}
In conclusione verrà redatto un documento che conterrà tutte le discussioni sugli obiettivi e le tecnologie utilizzate nel dettaglio per il raggiungimento dei risultati finali. 

